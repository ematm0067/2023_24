\documentclass[12pt]{article}
\usepackage{amsmath,amsfonts, epsfig}
\usepackage{booktabs} % for better table formatting
\usepackage[authoryear]{natbib}
\usepackage{array}
\usepackage{multirow}
\usepackage{graphicx}
\usepackage{fancyhdr}
\usepackage{bm}
\pagestyle{fancy}
\lfoot{\texttt{ematm0067.github.io} / \texttt{ematm0044.github.io}}
\lhead{Introduction to AI - 02.4\_logistic - Conor}
\rhead{\thepage}
\cfoot{}

\usepackage{tikz}
\usetikzlibrary{positioning}

\usepackage{ifthen}
\newboolean{nopics}
\setboolean{nopics}{true}


\begin{document}

\section*{Logistic regression.} 

Lets begin with an example, the Canadian linguist Jack Chambers is
interested in dialect and language change among English speakers in
the eastern part of Canada. He did a broad study where he discovered
many on-going changes and tried to interpret them. The data are
available, at least in summary form at
\texttt{dialect.topography.artsci.utoronto.ca/}. In one example, from
what is called the Golden Horseshoe, which extends along the western
shore of Lake Ontario and includes the cities of Hamilton and Ontario,
he discovered what to me looks like a surprising shift from weak to
strong participle for the word \textsl{sneak}. He asked participants which they say from
\begin{itemize}
\item The little devil sneaked into the theatre.
\item The little devil snuck into the theatre.
\end{itemize}
and roughly speaking found that younger people prefer `snuck' to
`sneaked', the so called strong form over the weak. I had assumed
there was a general trend towards weak forms, certainly my grandma used to say ``he was treat'.

This graph below gives a simulated version of his data; the actual
data have more points but bundle the participants into bands. For the
convenience of this lesson I have made artificial data, with fewer
points but where the ages take any value; the artificial data is
designed to have the same broad statistical structure as the real
data. In the graph age is plotted along the bottom, and the $y$-value
is zero for participants who say they use `snuck' and one for those
who use `sneaked'. A small jitter has been added to the $y$ value to
stop the points covering each other, this is a common and useful graphical device used for data like this.
\begin{center}
  \includegraphics[]{02.4_sneak.png}
  \end{center}

\section*{Summary}
In linear regression we assume the data come from a noisy version of a
linear process and we seek to model that process. It turns out there
is an analytic formula for the model parameters. This generalizes to
linear models that have lots of inputs and lots of predictions. The
error associated with the model can be minimimized by gradient flow,
since this is a solved problem it seems weird to do it approximately,
however numerical approaches like gradient descent works in other
cases where there is no analytic solution. We see how gradient on our
simple example.
\end{document}

