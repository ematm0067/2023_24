\documentclass[12pt]{article}
\usepackage{amsfonts, epsfig}
\usepackage{amsmath}
\usepackage{graphicx}
\usepackage{fancyhdr}
\pagestyle{fancy}
\lfoot{\texttt{emamtm0067.github.io} / \texttt{ematm0044.github.io}}
\lhead{Introduction to AI - Coursework}
\rhead{\thepage}
\cfoot{}
\begin{document}

\section*{Coursework }

\subsection*{Question 1 (40 pts)}

In this coursework you should create an artificial dataset and test
two classification algorithms on the data you have produced. The
dataset should be produced by first picking a number of dimensions, for
example $d=5$. Now select $c=3$ `center points' in that five
dimensional space, in the first instance pick each component uniformly
in the interval $[-1,1]$. Next for each center point, make $n=20$
datapoints from a multivariate normal distribution centered on the
center point with variance $\sigma^2$, with $\sigma=0.5$.

Now run two different classification algorithms on these data and
assess their accuracy. They should be pretty accurate. Consider three
different ways to change the simulated data to reduce their accuracy,
this could include changing $d$, $\sigma$, $n$. For high marks you
should include at least one more complicated manipulation, such as changing the
balance in the number of points for each center point or changing the
shape of the cloud of simulated data. For full marks this might
include cloud shapes that are nothing like a ball, not even the balls used
in Australian or American football.

Produce plots showing how the accuracy changes under these
manipulations and try to use your investigation to comment on the
relative advantages and disadvantages of the two algorithms you are
looking at. In looking at accuracy consider different measures of
accuracy and the accuracy of a baseline model. You should make sure
any assessment is not restricted to the data used in train models or
decide on hyper-parameters: it is important to hold aside testing
data.

In your report you should explain your decisions. You code will not be
marked for elegance, but it should run correctly; it is expected you
will use Python, but any of Python, Julia or R is fine. Do not include
screenshots of graphs, they should be imported directly; resize them
to the correct size before importing them, if the labels are tiny the
graphs will not be marked. Make sure figure captions are descriptive,
it is better to have some overlap between figure captions and the main
text than to have figure captions that are not reasonably
self-contained.

As a rough guide to marking:
\begin{itemize}
\item Initial production of simulated data, including some graphs or other approaches to visualisation. (6 marks).
\item Two algorithms should be tested along with three manipulations.
\item Overall presentation (3 marks), including use of appropriate
  sections, plots, diagrams, or tables to make your point. Do not
  include code snippets in the report. Instead, describe in words or
  equations what you are implementing. Format equations correctly.
\item Suitable choice of algorithms (2 marks).
\item Suitable choice of evaluation for algorithms (4 marks).
\item Two more obvious manipulations and an illustration of how they change classification accuracy, with a good and sensible set of measurements of accuracy. (2x5 marks)
\item A more advance manipulation of the simulated data along with the corresponding information on how this affects classification. (6 marks, this is reduced to 3 marks if the manipulation is another more obvious one).
\item Bonus marks if the advanced manipulation is very sophisticated. (2 marks)
\item Insight into what this experiments tells us about the two classification algorithm. (3 marks)
\item There are some marks (6 marks) for something surprising and
  unusual. This is to allow us to give credit for exceptional work,
  but you should not expend too much time chasing these final marks,
  concentrate on the standard requirements to get a good mark.
\end{itemize}

\subsection*{Question 2 (10 marks)}

New advances in AI will cause substantial changes in higher education,
in what we teach, in how we teach, how we examine. These changes come
with complex ethical challenges, for example, in how we assess
students in a fair way while teaching them to make use of AI tools. It
is already clear that students are using AI to do coursework
questions, particularly those with essay type answers. This makes the
marks or degree-grades less meaningful, narrowing the opportunity for
particularly able or hard-working students to succeed in competition
with students from more privileged backgrounds. Indeed, more
generally, AI may advantage some students over other, or may amplify
existing biases. More optimistically, AI may help address ethical
challenges already embedded in higher education. Discuss this.

\subsection*{Report}

Your report should be no longer than five pages, excluding any
references. It is expected that Question 2 would occupy about a fifth
of this space; use an 11 or 12pt font and do not try tricks like
expanding the margin to fit in more text, shorter is better than
longer.

Your report must be submitted in pdf and should be prepared in LaTeX;
overleaf is a good approach, but not required as long as LaTeX has
been used\footnote{R-markdown and some other notebook-based environments typeset using LaTeX, this is acceptable}. As always when using LaTeX, give yourself over to defaults,
our expectation of what a document should look like has been
conditioned on LaTeX, so it is best not to try to override the look of
the document. I have included a template but you need not use that.

Avoid code snippets in the report unless that feels like the best way
to illustrate some subtle aspect of an algorithm; do always though
consider a mathematical description if possible. You will be asked to
submit code and it may be tested to make sure it works and matches
your report. It will not, however, be marked in and of itself.


\subsection*{Submission}

The deadline for report and code: 13h00 (GMT+1) on 2024-08-05, there
will be a submission point on Blackboard under the ``assessment,
submission and feedback'' link. Please upload the following two files:
\begin{enumerate}
\item Your report as a PDF with filename [student\_number].pdf, where ``[student\_number]'' is replaced by your student number, not your username. Upload this to the submission point ``Introduction to AI Coursework (Turnitin)''.
\item Your code inside a single zip file with filename
  [student\_number].zip. Inside the zip file there should be a single
  folder containing your code, with your student number as the folder
  name. Please remove datasets and other large files to minimise the
  upload size - we only need the code itself. Upload this file to the
  submission point ``Code for Introduction to AI Coursework''.
\end{enumerate}
  
We may review your Python code by eye but your marks will be based on
the contents of your report, with the code used to check how you
carried out the experiments described in your report. We will not give
marks for the coding style, comments, or organisation of the
code. Code written in Julia or R is also acceptable as is the use of a
standard notebook format. If you are particularly keen on another
programming language let me know and I will consider this; I would
accept other modern languages such as Rust, but outmoded or unsuitable
languages like C++, Java or MATLAB would not be allowed.

Please do not include your name in the report text itself: to ensure fairness, we mark the reports anonymously.

Avoiding Academic Offences: Please re-read the university's plagiarism
rules to make sure you do not break any rules. Academic offences
include submission of work that is not your own, falsification of data
/ evidence or the use of materials without appropriate
referencing. Note that sharing your report with others is also not
allowed. These offences are all taken very seriously by the University
and we have very little leeway within the framework the University has
set out. Do not copy text directly from your sources - always rewrite
in your own words and provide a citation. Work independently -- do not
share your code or reports with others; you can, of course, discuss
your work with your classmates, but do not share text or code.

Suspected offences will be dealt with in accordance with the
University's policies and procedures. If an academic offence is
suspected in your work, you will be asked to attend an interview with
senior members of the school, where you will be given the opportunity
to defend your work. The plagiarism panel can apply a range of
penalties, depending on the severity of the offence. These include a
requirement to resubmit work, capping of grades and the award of no
mark for an element of assessment. Again, we are not in a position to
be lenient here, the academic offences procedure is not one we control.

\subsection*{Extensions and Exceptional Circumstances}

If the completion of your assignment has been significantly disrupted
by serious health conditions or personal problems, or other serious
issues, you can apply for consideration in accordance with the normal
university policy and processes. Students should refer to the guidance
and complete the application forms as soon as possible when the
problem occurs. Please see the guidance below and discuss with your
personal tutor for more advice:
\begin{itemize}
\item \texttt{www.bristol.ac.uk/students/support/academic-advice/\\
  assessment-support/request-a-coursework-extension/}
\item \texttt{www.bristol.ac.uk/students/support/academic-advice/\\assessment-support/exceptional-circumstances/}
    \end{itemize}



\end{document}
