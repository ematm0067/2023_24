\documentclass[12pt]{article}
\usepackage{amsfonts, epsfig}
\usepackage{amsmath}
\usepackage{graphicx}
\usepackage{fancyhdr}
\pagestyle{fancy}
\lfoot{\texttt{emamtm0067.github.io} / \texttt{ematm0044.github.io}}
\lhead{Introduction to AI - Coursework}
\rhead{\thepage}
\cfoot{}
\begin{document}

\section*{Coursework }

\subsection*{Question 1 (40 pts)}

Please download the \texttt{penguin.csv} dataset. This dataset has
information about penguins of one of three types. You task is it
explore the dataset and to predict the penguin type. The dataset is
known as the Palmer Penguins and is found at:\\
\\
\texttt{allisonhorst.github.io/palmerpenguins/}\\
\\
This link contains information on how to cite the dataset.

You should consider how to visualize the data and which algorithms to
try. Nothing you do will be completely successful, this coursework is
not here to judge your final accuracy but the care you bring to your
investigation. Here are some thing you should consider:
\begin{itemize}
  \item The kind of algorithm to use, for example whether to classify, regress  or cluster.
  \item The metric to use to measure the performance of the model.
  \item What sort of baseline to compare the model to.
  \item How to choose the hyperparameters of your model.
\end{itemize}
For good marks you should include some graphs that illustrate
properties of the data and some exploration based on either
unsupervised learning or regression, along with that you should
compare two classification algorithms, both to each other and to a
baseline model. The baseline model is just the performance you would
get if you guessed without reference to anything but the sizes of the
populations in each class. The algorithms you pick do not need to be
unusual, for example $k$nn classification would be perfectly good,
though, of course, for full marks this would include some
consideration of how to pick $k$ and how to measure the distance, even
though, as you know, no approach to choosing $k$ is ever going to be
completely satisfactory. In addition, you should include either some
exploratory regression or unsupervised learning; for regression you
might regress two properties and examine whether the regression
parameters are the same for each penguin type; unsupervised learning
could use $k$-means, for example. You do not need to do both
regression and unsupervised learning.

Thus, there are four elements expected:
\begin{enumerate}
\item A brief exploration of the data.
\item An unsupervised or regression approach.
\item A classification algorithm.
\item Another classification algorithm,
\end{enumerate}
and, in the case of the (3) and (4), it is expected that you will
compare the two classification approaches. More detail is provided by
the marking scheme below.

You should make sure any assessment is not restricted to the data used
in train models or decide on hyper-parameters: it is important to hold
aside testing data. In your report you should explain your
decisions. You code will not be marked for elegance, but it should run
correctly; it is expected you will use Python, but any of Python,
Julia or R is fine. Do not include screenshots of graphs, they should
be imported directly; resize them to the correct size before importing
them, if the labels are tiny the graphs will not be marked. Make sure
figure captions are descriptive, it is better to have some overlap
between figure captions and the main text than to have figure captions
that are not reasonably self-contained.

As a rough guide to marking:
\begin{itemize}
\item Initial description of the data, including some graphs or other approaches to visualisation. 6 marks.
\item Either unsupervised learning or regression. 6 marks.
\item Two algorithms should be tested, if only one algorithm is
  included the 28 available marks will be halved.
\item Overall presentation (3 marks), including use of appropriate
  sections, plots, diagrams, or tables to make your point. Do not
  include code snippets in the report. Instead, describe in words or
  equations what you are implementing. Format equations correctly.
\item Suitable choice of algorithms (3 marks).
\item Suitable choice of evaluation for algorithms (3 marks).
\item Comparison with a suitable baseline (3 marks) and a justification for which baseline to use.
\item A description of metaparameter selection (4 marks), if one
  algorithm has not metaparameter, then explain that and note why not
  and why this do or does not make it a better algorithm for these
  data.
\item Describe and compare the results from your two classification
  algorithms, include a description of how you implemented the
  algorithms. (6 marks)
\item There are some marks (6 marks) for something suprising and
  unusual. This is to allow us to give credit for exceptional work,
  but you should not expend too much time chasing these final marks,
  concentrate on the stardard requirements to get a good mark.
\end{itemize}

\subsection*{Question 2 (10 marks)}

For two of these three types of ethical challenge or threat facing us
in data science and AI:
\begin{enumerate}
\item The protection of data, of the people whose data they are and participants in any study.
\item Avoiding the amplification of biases and regressive values implicit in historic dataset.
\item The safety of AI systems and the possible of existential threats from machines.
\end{enumerate}
describe what you think is a specific example that could arise or has
arisen in the past; this could be a single example that relates to the
two challenges you are considering, or it could be two examples, one
for each challenge. Obviously the three broad types of challenge
overlap, do not worry about the boundaries between these types, but do
try to address different two different types of threat in your
examples. Explain how the ethical problems could be addressed, or at
least made more transparent.

\subsection*{Report}

Your report should be no longer than five pages, excluding any
references. It is expected that Question 2 would occupy about a fifth
of this space; use an 11 or 12pt font and do not try tricks like
expanding the margin to fit in more text, shorter is better than
longer.

Your report must be submitted in pdf and should be prepared in LaTeX;
overleaf is a good approach, but not required as long as LaTeX has
been used\footnote{R-markdown and some other notebook-based environments typeset using LaTeX, this is acceptable}. As always when using LaTeX, give yourself over to defaults,
our expectation of what a document should look like has been
conditioned on LaTeX, so it is best not to try to override the look of
the document. I have included a template but you need not use that.

Avoid code snippets in the report unless that feels like the best way
to illustrate some subtle aspect of an algorithm; do always though
consider a mathematical description if possible. You will be asked to
submit code and it may be tested to make sure it works and matches
your report. It will not, however, be marked in and of itself.


\subsection*{Submission}

The deadline for report and code: 13h00 (GMT+1) on 22nd May, there
will be a submission point on Blackboard under the ``assessment,
submission and feedback'' link. Please upload the following two files:
\begin{enumerate}
\item Your report as a PDF with filename <student\_number>.pdf, where ``<student\_number'' is replaced by your student number, not your username. Upload this to the submission point ``Introduction to AI Coursework (Turnitin)''.
\item Your code inside a single zip file with filename
  <student\_number>.zip. Inside the zip file there should be a single
  folder containing your code, with your student number as the folder
  name. Please remove datasets and other large files to minimise the
  upload size - we only need the code itself. Upload this file to the
  submission point ``Code for Introduction to AI Coursework''.
\end{enumerate}
  
We may review your Python code by eye but your marks will be based on
the contents of your report, with the code used to check how you
carried out the experiments described in your report. We will not give
marks for the coding style, comments, or organisation of the code.

Please do not include your name in the report text itself: to ensure fairness, we mark the reports anonymously.

Avoiding Academic Offences: Please re-read the university's plagiarism
rules to make sure you do not break any rules. Academic offences
include submission of work that is not your own, falsification of data
/ evidence or the use of materials without appropriate
referencing. Note that sharing your report with others is also not
allowed. These offences are all taken very seriously by the University
and we have very little leeway within the framework the University has
set out. Do not copy text directly from your sources - always rewrite
in your own words and provide a citation. Work independently -- do not
share your code or reports with others; you can, of course, discuss
your work with your classmates, but do not share text or code.

Suspected offences will be dealt with in accordance with the
University's policies and procedures. If an academic offence is
suspected in your work, you will be asked to attend an interview with
senior members of the school, where you will be given the opportunity
to defend your work. The plagiarism panel can apply a range of
penalties, depending on the severity of the offence. These include a
requirement to resubmit work, capping of grades and the award of no
mark for an element of assessment. Again, we are not in a position to
be lenient here, the academic offences procedure is not one we control.

\subsection*{Extensions and Exceptional Circumstances}

If the completion of your assignment has been significantly disrupted
by serious health conditions or personal problems, or other serious
issues, you can apply for consideration in accordance with the normal
university policy and processes. Students should refer to the guidance
and complete the application forms as soon as possible when the
problem occurs. Please see the guidance below and discuss with your
personal tutor for more advice:
\begin{itemize}
\item \texttt{www.bristol.ac.uk/students/support/academic-advice/\\
  assessment-support/request-a-coursework-extension/}
\item \texttt{www.bristol.ac.uk/students/support/academic-advice/\\assessment-support/exceptional-circumstances/}
    \end{itemize}



\end{document}
