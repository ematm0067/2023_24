\documentclass[12pt]{article}
\usepackage{amsfonts, epsfig}
\usepackage{amsmath}
\usepackage{graphicx}
\usepackage{fancyhdr}
\pagestyle{fancy}
\lfoot{\texttt{emamtm0067.github.io}}
\lhead{Introduction to AI - Coursework}
\rhead{\thepage}
\cfoot{}
\begin{document}

\section*{Coursework }

\subsection*{Question 1 (40 pts)}

Please download the \texttt{penguin.csv} dataset. This dataset has
information about penguins of one of three types. You task is it
explore the dataset and to predict the penguin type.

You should consider how to visualize the data and which algorithms to
try. Nothing you do will be completely successful, this coursework is
not here to judge your final accuracy but the care you bring to your
investigation. Here are some thing you should consider:
\begin{itemize}
  \item The kind of algorithm to use, for example whether to classify, regress  or cluster.
  \item The metric to use to measure the performance of the model.
  \item What sort of baseline to compare the model to.
  \item How to choose the hyperparameters of your model.
\end{itemize}
For good marks you should include some graphs that illustrate
properties of the data and you should compare two algorithms, both to
each other and to a baseline model. You should make sure any
assessment is not restricted to the data used in train models or
decide on metaparameters. In your report you should explain your
decisions. You code will not be marked for elegance, but it should run
correctly; it is expected you will use Python, but any of Python,
Julia or R is fine. Do not include screenshots of graphs, they should
be imported directly; resize them to the correct size before importing
them, if the labels are tiny the graphs will not be marked.

As a rough guide to marking:
\begin{itemize}
\item Initial description of the data, including some graphs or other approaches to visualisation. 8 marks.
\item Two algorithms should be tested, if only one algorithm is
  included the 32 available marks will be halved.
\item Overall presentation (4 marks), including use of appropriate
  sections, plots, diagrams, or tables to make your point. Do not
  include code snippets in the report. Instead, describe in words or
  equations what you are implementing. Format equations correctly.
\item Suitable choice of algorithms (6 marks).
\item Suitable choice of evaluation for algorithms (3 marks).
\item Comparison with a suitable baseline (3 marks) and a justification for which baseline to use.
\item A description of metaparameter selection (3 marks), if one
  algorithm has not metaparameter, then explain that and note why not
  and why this do or does not make it a better algorithm for these
  data.
\item Describe and compare the results from your two algorithms,
  include a description of how you implemented the algorithms. (10 marks)
\item There are some marks (3 marks) for something suprising and unusual.
\end{itemize}

\subsection*{Question 2 (10 marks)}

For two of these three types of ethical challenge facing us in data science and AI:
\begin{enumerate}
\item The protection of data, of the people whose data they are and participants in any study.
\item Avoiding the amplification of biases and regressive values implicit in historic dataset.
\item The safety of AI systems and the possible of existential threats from machines.
\end{enumerate}
describe what you think is a specific example of a challenge that
could arise or has arisen in the past. Obviously the three broad types
of challenge overlap, do not worry about the boundaries between these
types, but do try to address different types of threat in your
examples. Explain how the ethical problems could be addressed, or at
least made more transparent.

\subsection*{Report}

Your report should be no longer than five pages, including any
references. It is expected that Question 2 would occupy about a fifth
of this space; use an 11 or 12pt font and do not try tricks like
expanding the margin to fit in more text, shorter is better than
longer.

Your report must be submitted in pdf and should be prepared in LaTeX;
overleaf is a good approach, but not required as long as LaTeX has
been used. As always when using LaTeX, give yourself over to defaults,
our expectation of what a document should look like has been
conditioned on LaTeX, so it is best not to try to override the look of
the document.

\end{document}
